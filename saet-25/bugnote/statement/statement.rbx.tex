%- block legend

Ricardo Yagami é um genial professor universitário de algoritmos e estruturas de dados, porém ele está entediado do jeito que anda o mundo.  
Os dias se passam e as mesmas notícias: alunos usando IA para fazer trabalhos de algoritmos, agradecendo ao ChatGPT ao invés de Alan Turing — algo realmente deplorável.

Porém, a vida de Ricardo mudou em certo dia, quando ele estava em sua sala e, de repente, viu um caderno preto caindo do céu pela janela.  

Como já estava entediado, foi até o caderno para ver do que se tratava e começou a ler:

\begin{center}
\textbf{BugNote (Caderno do Bug)}\\
\end{center}

\begin{quote}
Modo de usar:
\begin{itemize}
  \item A pessoa cujo primeiro nome for escrito neste caderno terá seu código bugado.
  \item Após escrever o primeiro nome da pessoa, deve-se especificar a linha exata do código dela onde o bug ocorrerá.
  \item O caderno não surtirá efeito se o nome da pessoa for escrito errado.
  \item O caderno não surtirá efeito se a linha do código escrita nele exceder o total de linhas no código real.
  \item Uma pessoa cuja a regra anterior vier a ocorrer poderá ter seu nome escrito novamente e ainda ter efeito, porém se o número de tentativas erradas for maior ou igual a $3$, a pessoa passa a ser imune para sempre aos efeitos do BugNote.
\end{itemize}
\end{quote}

Após testar o caderno em seu próprio código, Ricardo confirmou sua veracidade e decidiu bolar um plano: punir todos os alunos que usam IA para fazer código ao invés de aprender — e se tornar o Sênior do novo mundo!

Dado o número de alunos e o número de nomes que Ricardo escreveu no BugNote, determine quais alunos foram punidos.
%- endblock

%- block input
 primeira linha contém dois inteiros $N$ e $Q$ ($1 \leq N, Q \leq 2000$), representando respectivamente o número de alunos na turma e o número de nomes que Ricardo escreveu no BugNote.

As próximas $N$ linhas contêm uma string $s_i$ (o nome do aluno, único) e um inteiro $l_i$ ($1 \leq l_i \leq 10^5$), representando o número de linhas no código daquele aluno.

Em seguida, as próximas $Q$ linhas contêm uma string $t_i$ e um inteiro $x_i$, representando o nome e a linha do código que Ricardo escreveu no BugNote.
%- endblock

%- block output
A saída deve conter um inteiro $K$ representando o número de alunos punidos seguido de $K$ linhas com os nomes desses alunos.

Caso nenhum aluno tenha sido punido, imprima \texttt{-1}.
%- endblock

%- block editorial
This is the editorial.
%- endblock
