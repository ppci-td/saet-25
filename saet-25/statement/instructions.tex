% filler para alinhar os problemas com as paginas impressas
\section*{Olá!}

Prova preparada com muito empenho por \textbf{Henrique Farias} e \textbf{Ricardo
    Oliveira} para a Semana Acadêmica de Engenharia e Tecnologia (SAET) 2025 da
    UTFPR-Toledo. Esperamos que gostem!\\

\begin{center}
\textit{Good luck \& Have Fun}
\end{center}

\newpage

\section*{Informações Gerais}

Esta prova contém \pageref*{lastpage} páginas (excluindo a capa), numeradas de 1 a \pageref*{lastpage}. Verifique se a prova está completa.

\subsection*{Entrada}

\begin{itemize}
   \item A entrada deve ser lida da entrada padrão.

	\item Todas as linhas da entrada, incluindo a última, terminam com o caractere de fim-de-linha (\texttt{\textbackslash n}).

	\item Quando a entrada contém vários valores separados por espaço, existe exatamente um espaço entre dois valores consecutivos na mesma linha.

\end{itemize}

\subsection*{Saida}

\begin{itemize}

    \item A saida deve ser escrita na saida padrão.

    \item A saida deve ser impressa no formato especificado pelo problema. A saida não deve conter nenhum dado adicional.

	\item Todas as linhas da saida, incluindo a última, terminam com o caractere de fim-de-linha (\texttt{\textbackslash n}).

    % Geradores atuais podem passar por isso, nao sendo sempre necessario
	%\item When a line in the output displays multiple space-separated values, there must be exactly one whitespace character between two consecutive values.
	%\item When an output value is a real number, use at least the number of decimal places corresponding to the precision required in the problem statement.
\end{itemize}

% \subsection*{Interactive Problems}

% The contest may contain interactive problems. In this type of problem, the input data provided to your program may not be predetermined but is instead specifically constructed for your solution. The judge writes a special program (the interactor), whose output is transferred to your solution's input, and your program's output is sent to the interactor's input. In other words, your solution and the interactor exchange data and may decide what to print based on the communication history.

% When writing a solution for an interactive problem, it is important to remember that if you print any data, it may first be stored in an internal \textit{buffer} and not immediately transferred to the interactor. To avoid this situation, you must use a special \textit{flush} operation every time you print data. These flush operations are available in the standard libraries of almost all programming languages:

% \begin{itemize}
% 	\item \texttt{fflush(stdout)} in \texttt{C}.
% 	\item \texttt{cout.flush()} in \texttt{C++}.
% 	\item \texttt{sys.stdout.flush()} in \texttt{Python}.
% 	\item \texttt{System.out.flush()} in \texttt{Java}.
% \end{itemize}
