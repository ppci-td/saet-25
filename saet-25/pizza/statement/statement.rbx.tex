%- block legend
Nada como uma agradável noite com seus amigos na sua casa comendo uma pizza
enquanto assistem seu \textit{anime} favorito!

Opa, a pizza chegou! Como o grupo tem 4 pessoas ao todo, você decidiu cortar a
pizza em 4 pedaços. Você fez um corte horizontal e outro corte vertical, de tal
forma que os cortes intersectam em um ponto $P$ na pizza.

Oh não! Você acabou de perceber que os 4 pedaços podem ter ficado de tamanho
diferentes! Dados o raio da pizza e a coordenada do ponto $P$, determine a área
dos 4 pedaços obtidos.

%- endblock

%- block input
A entrada contém uma linha com três inteiros $R$, $X$ e $Y$
($1 \leq R \leq 30, \sqrt{X^2 + Y^2} < R$), o raio da pizza e as coordenadas do ponto
$P$, respectivamente. Considere que a pizza está centrada na origem (isto é, seu
centro é o ponto $(0,0)$).

%- endblock

%- block output
Imprima uma linha com quatro valores $A_1$, $A_2$, $A_3$ e $A_4$ indicando a área de cada
pedaço da pizza, arredondadas com três casas decimais. Imprima os valores em
ordem não-decrescente (isto é, de forma que $A_1 \leq A_2 \leq A_3 \leq A_4$).
%- endblock

%- block notes
A figura abaixo representa o primeiro exemplo de entrada:
\begin{center}
    \includegraphics[scale=1.25]{pizza.png}
\end{center}
%- endblock

%- block editorial
Primeiramente, faremos $X=|X|$ e $Y=|Y|$ de forma a colocar o ponto $P$ no
primeiro quadrante do círculo. Por simetria isto não altera a resposta, e nos
permite considerar apenas o caso em que $P$ está no primeiro quadrante.

Cada área pode então ser calculada separadamente, de maneira analítica:

\begin{center}
\begin{multicols}{2}
\includegraphics[scale=1.2]{A1.png}\\
Área = $\angle OAB + \vartriangle ACO + \vartriangle OBD + \square OCPD$\\

\includegraphics[scale=1.2]{A2.png}\\
Área = $\angle OAB + \vartriangle BCP - \vartriangle OAC$\\
\end{multicols}
\begin{multicols}{2}
\includegraphics[scale=1.2]{A3.png}\\
Área = $\angle OAB - \vartriangle OAP - \vartriangle OBC$\\

\includegraphics[scale=1.2]{A4.png}\\
Área = $\angle OAB + \vartriangle BCP - \vartriangle ACD$\\
\end{multicols}
\end{center}

Alternativamente, quando três das quatro áreas são calculadas, a quarta pode ser
dada pela subtração da área total ($\pi R^2$) das demais áreas.

Lembre de sempre levar consigo uma boa implementação de funções da geometria!

Complexidade: $O(1)$

%- endblock
