%- block legend
Uma \textit{string} de abre e fecha parênteses está \textit{bem balanceada} se:
ou é $($ seguido de uma \textit{string} bem balanceada seguida de $)$;
ou é a concatenação de duas \textit{strings} bem balanceadas;
ou é vazia (formalmente, é uma \textit{string} gerada pela gramática $S \rightarrow (S) | SS | \varepsilon$).

É dada uma \textit{string} inicial $s$ com $N$ parênteses. Processe $Q$ operações, onde cada operação pode ser:
\begin{itemize}
    \item $1$ $l$ $r$: inverta todos os parênteses no intervalo
    $[l..r]$; isto é,
    para todo $l \leq i \leq r$, troque $s[i]$ pelo inverso de $s[i]$. O
    \textit{inverso} de $)$ é $($, enquanto o inverso de $($ é $)$;
    \item $2$ $l$ $r$: determine se a \textit{substring} $s[l..r]$ está bem balanceada.
\end{itemize}

%- endblock

%- block input
A primeira linha contém dois inteiros $N$ e $Q$
($1 \leq N, Q \leq \VAR{vars.N.max | sci}$), o tamanho da \textit{string} e o
número de operações.
A segunda linha contém a \textit{string} inicial com $N$ parênteses.
As próximas $Q$ linhas descrevem uma operação cada, na forma
$t$ $l$ $r$ onde $t=1$ ou $t=2$, e $1 \leq l \leq r \leq N$.

%- endblock

%- block output
Para cada operação com $t=2$, imprima uma linha com \texttt{sim} se a \textit{substring}
$s[l..r]$ está bem balanceada, ou \texttt{nao} caso contrário.
%- endblock


%- block notes
Considere o primeiro exemplo dado.
Na primeira operação, a resposta é \texttt{sim} porque
$s[6..9] = $\texttt{()()} está bem balanceada. Na segunda operação, a resposta é
\texttt{nao} porque $s[1..10]$ = \texttt{()())()())} não está bem balanceada.

Após a terceira operação (inverte $s[4..6]$), a \textit{string} se torna
\texttt{()((())())}. A resposta da quarta operação é \texttt{sim} porque agora
$s[1..10]$= \texttt{()((())())} está bem balanceada.

%- endblock

%- block editorial
This is the editorial.
%- endblock
