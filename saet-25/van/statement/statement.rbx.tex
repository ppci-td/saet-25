%- block legend
Está chegando a hora! A etapa regional da Maração de Programatona ocorrerá nas
próximas semanas, e o seu \textit{coach} já está vendo como irá levar todos os $N$
participantes da universidade para competir na cidade vizinha.

A universidade disponibiliza uma van para essas viagens. A van conta com $M$
poltronas (além da do motorista), de diferentes larguras. Algumas poltronas são tão pequenas e
apertadas que infelizmente não é possível usá-las na viagem; as poltronas que
não pode ser usadas são aquelas com $L$ cm ou menos de largura.

Sua tarefa é ajudar o seu \textit{coach} a determinar se é possível levar todos
os $N$ participantes na van (em viagem única), ou se será necessário usar outros veículos além da van.

%- endblock

%- block input
A primeira linha contém dois inteiros $N$ e $M$ ($1 \leq N, M \leq \VAR{vars.N.max | sci}$), o número de competidores e de poltronas na van.
A próxima linha contém $M$ inteiros $l_i$ ($1 \leq l_i \leq \VAR{vars.li.max | sci}$)
indicando a largura de cada poltrona, em centímetros.
A última linha contém um inteiro $L$ ($1 \leq L \leq \VAR{vars.L.max | sci}$)
indicando que poltronas com $L$ cm ou menos de largura não podem ser usadas.
%- endblock

%- block output
Imprima uma linha com
\texttt{SIM} se é possível levar todos os competidores na
van, ou \texttt{NAO} caso contrário.
%- endblock

%- block editorial
Para cada poltrona, verifique se sua largura é igual ou menor a $L$ (isto é, se
        $l_i \leq L$). Se for, incremente um contador.
Ao final da verificação, este contador terá o número de poltronas que podem ser
usadas. A resposta é \texttt{SIM} se e somente se este número for maior ou
igual a $N$.

Complexidade: $O(M)$
%- endblock
