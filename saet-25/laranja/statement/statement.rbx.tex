%- block legend
Hicard adora laranjas, tanto que todo dia no RU (Restaurante Universitário) ele sempre pega uma laranja de sobremesa e leva para casa para poder saciar-se mais tarde.  
Porém, uma laranja por dia não é o bastante para ele. Por isso, seus amigos da faculdade sempre pegam a laranja a que têm direito e doam para que Hicard possa levar mais de uma laranja para casa.

Além de ser um grande fã de laranjas, Hicard também é apaixonado por análise combinatória.
Considere que Hicard tem inicialmente $0$ laranjas. Então, após cada dia $i$, ele ganha $G_i$ novas laranjas e então come $C_i$ laranjas.

Ele gostaria de saber, ao final de cada dia, de quantas maneiras pode rearranjar as laranjas que ainda tem em sua fruteira, que possui exatamente $3$ gavetas, cada uma podendo conter até $4$ laranjas. Tanto as laranjas quanto as gavetas são distinguíveis entre si, mas, dentro de uma gaveta, a ordem em que as laranjas ficam é irrelevante.

%- endblock

%- block input
A primeira linha contém um inteiro $N$ ($1 \le N \le 10^4$), o número de dias que Hicard pegou ou comeu laranjas.

As próximas $N$ linhas contêm dois inteiros $G_i$ e $C_i$ ($0 \le G, C \le 100$), representando o número de laranjas que Hicard ganhou e comeu em cada dia, respectivamente.
%- endblock

%- block output

A saída deve conter $N$ linhas.  
Em cada linha, imprima o número de maneiras diferentes que Hicard pode organizar sua fruteira ao final do respectivo dia.
Se houver mais laranjas do que espaço na fruteira em determinado dia, isto é, se Hicard ter mais que $12$ laranjas no final de um dia, imprima $-1$ para aquele dia.
É garantido que que o número de laranjas ao final de um dia nunca será negativo.
%- endblock

%- block notes
No exemplo de entrada, Hicard tem 2 laranjas ao final do primeiro dia (por exemplo, a laranja $A$ e a $B$). As 9 maneiras de rearranjá-las nas 3 gavetas são
$(\{A,B\},\{\},\{\})$, $(\{\},\{A,B\},\{\})$, $(\{\},\{\},\{A,B\})$,
$(\{A\},\{B\},\{\})$, $(\{A\},\{\},\{B\})$, $(\{B\},\{A\},\{\})$,
$(\{B\},\{\},\{A\})$, $(\{\},\{A\},\{B\})$ e $(\{\},\{B\},\{A\})$.
%- endblock

%- block editorial
Para a solução deste problema primeiro levamos em consideração que tanto as laranjas quanto as gavetas são distinguíveis, o problema equivale a contar o número de maneiras de particionar \(l\) elementos distintos em três subconjuntos ordenados \((A, B, C)\) tais que \(|A|, |B|, |C| \leq 4\) e \(|A| + |B| + |C| = l\).  
Para cada possível par de tamanhos \((j, k)\) das duas primeiras gavetas, o tamanho da terceira gaveta é \(cur = l - j - k\). Se \(cur\) for válido (entre 0 e 4), o número de maneiras de escolher quais laranjas vão para cada gaveta é dado pelo número multinomial:
\[
\frac{l!}{j! \, k! \, cur!}.
\]
Assim, somando sobre todas as combinações válidas de \((j, k)\), obtemos o número total de arranjos possíveis. 

A solução esperada pré-calcula os fatoriais de \(0!\) até \(12!\) para permitir o cálculo eficiente das combinações multinomiais. Para cada dia, ele itera sobre todos os pares de tamanhos de gavetas \((j, k)\) de 0 a 4 e soma os resultados válidos. Por fim, imprime o número total de maneiras correspondentes ou \(-1\) se houver excesso de laranjas.

Complexidade total: $O(N)$

%- endblock
