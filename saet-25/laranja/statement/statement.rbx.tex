%- block legend
Hicard adora laranjas, tanto que todo dia no RU (Restaurante Universitário) ele sempre pega uma laranja de sobremesa e leva para casa para poder saciar-se mais tarde.  
Porém, uma laranja por dia não é o bastante para ele. Por isso, seus amigos da faculdade sempre pegam a laranja a que têm direito e doam para que Hicard possa levar mais de uma laranja para casa.

Além de ser um grande fã de laranjas, Hicard também é apaixonado por análise combinatória, após cada dia, ele ganha $G$ novas laranjas e come $C$ laranjas.  
Ele gostaria de saber de quantas maneiras pode rearranjar, ao final de cada dia, as laranjas em sua fruteira, que possui exatamente $3$ gavetas, cada uma podendo conter até $4$ laranjas.
%- endblock

%- block input
A primeira linha contém um inteiro $N$ ($1 \le N \le 10^4$), o número de dias que Hicard pegou ou comeu laranjas.

As próximas $N$ linhas contêm dois inteiros $G$ e $C$ ($0 \le G, C \le 100$), representando o número de laranjas que Hicard ganhou e comeu em cada dia, respectivamente.
Considere que Hicard tenha inicialmente $0$ laranjas.
%- endblock

%- block output

A saída deve conter $N$ linhas.  
Em cada linha, imprima o número de maneiras diferentes que Hicard pode organizar sua fruteira ao final do respectivo dia.
Se houver mais laranjas do que espaço na fruteira em determinado dia, isto é, se Hicard ter mais que $12$ laranjas no final de um dia, imprima $-1$ para aquele dia.
É garantido que que o número de laranjas nunca será negativo e que a ordem das laranjas dentro das fruteiras é irrelevante.
%- endblock

%- block editorial
This is the editorial.
%- endblock
