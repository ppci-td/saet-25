%- block legend
"O Jogo" é um jogo bem peculiar de se jogar, onde basicamente você só está ganhando quando esquece completamente que ele existe, e está perdendo enquanto está consciente do jogo (Logo, você leitor está perdendo o jogo neste exato momento).  

Erick é um cara muito organizado, e decidiu anotar em um caderno todas as datas em que se lembrou da existência do jogo e, consequentemente, perdeu. Agora, ele quer saber, dado esse histórico, qual foi o menor e o maior intervalo de tempo (em dias) que conseguiu ficar sem perder o jogo.
%- endblock

%- block input
A primeira linha da entrada contém um inteiro $N$ ($3 \leq N \leq 10^5$), representando o número de datas registradas.  

Cada uma das próximas $n$ linhas contém três inteiros $D$, $M$, $A$ ($1 \leq D \leq 30$, $1 \leq M \leq 12$, $1 \leq A \leq 10^9$), representando o dia, o mês e o ano de uma data em que Erick perdeu o jogo. É garantido que as datas estão em ordem cronológicas.

Considere que todos os meses tem $30$ dias e todos os anos $360$ dias;
%- endblock

%- block output
A saída deve conter dois inteiros: o menor intervalo e o maior intervalo, em dias, entre duas datas consecutivas em que Erick perdeu o jogo.
%- endblock

%- block editorial
Partindo da segunda data podemos calcular a distância de tempo em relação a data anterior da seguinte forma: $(a_i - a_{i-1}) \times 360 + (m_i - m_{i-1}) \times 30 + (d_i - d_{i-1})$.
Para cada iteração pode-se tirar o mínimo e máximo dos valores e no final imprimi-los como resposta correta.

%- endblock
