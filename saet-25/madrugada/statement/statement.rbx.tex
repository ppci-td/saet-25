%- block legend
Henrique ``Gubi'' é um garoto que adora Maratonas de Programação! Certo noite, depois de uma longa e divertida Maratona, Gubi e seus amigos (que também são maratonistas) tiveram uma ideia: passar toda a madrugada acordados na praia, curtindo a areia, o mar e o luar.  Gubi topou na hora! Afinal, ele já está acostumado a virar as noites acordado, e agora fará isso com seus amigos em um lugar \textit{good vibes}.

Gubi e seus amigos decidiram levar para a praia todos os $B$ balões que conquistaram na Maratona daquele dia. Em um determinado momento da noite, entretanto, uma forte ventania levou $V$ desses balões embora.

No final da noite (já ao amanhecer), Gubi e seus amigos resolveram dividir \textit{todos} os balões que sobraram entre as $N$ pessoas do grupo, de forma que todas as pessoas ficassem com a mesma quantidade de balões.

Ajude Gubi a determinar se é possível dividir todos os balões que sobraram igualmente entre as pessoas do grupo e, se sim, com quantos balões cada pessoa vai ficar.

%- endblock

%- block input
A entrada contém três inteiros $N$, $B$ e $V$
($1 \leq N \leq \VAR{vars.N.max | sci}$, $1 \leq B \leq \VAR{vars.B.max}$, $0 \leq V \leq B$)
indicando o número de pessoas no grupo, o número total de balões inicialmente na
praia e o número de balões levados pelo vento, respectivamente.
%- endblock

%- block output
Imprima uma linha com o número de balões com que cada pessoa irá ficar. Se não
for possível fazer a divisão, imprima \texttt{-1}.
%- endblock

%- block editorial
Sobraram $B-V$ balões após a ventania. Assim, verifique (com \texttt{if}) se
$B-V$ é divisível por $N$, testando se o resto da divisão é igual a 0 (isto é,
        se $(B-V) \% N = 0$). Se for, imprima $\displaystyle\frac{B-V}{N}$. Caso contrário,
        imprima $-1$.

Complexidade: $O(1)$
%- endblock
