%- block legend
A \textit{Respiração do Trovão} é uma técnica de luta milenar usada em combate
por grandes guerreiros. Existem $F$ \textit{formas} diferentes de usar a técnica. Cada
forma exige muito para ser dominada; por isso, alguns guerreiros
podem dominar apenas algumas das formas, enquanto não dominem outras.

O Vovô é um grande mestre da Respiração do Trovão, e tem hoje $N$ discípulos sob
seu treinamento. O Vovô sabe quais formas cada discípulo seu domina, e agora quer escolher dois de seus discípulos para criar a dupla perfeita. Uma dupla é perfeita se cada forma da Respiração do Trovão é dominada por
pelo menos um discípulo da dupla.

De quantas maneiras o vovô pode escolher dois discípulos para criar a dupla
perfeita?

%- endblock

%- block input
A primeira linha contém dois inteiros $N$ e $F$ ($\VAR{vars.N.min} \leq N \leq
\VAR{vars.N.max |sci}, \VAR{vars.F.min} \leq F \leq \VAR{vars.F.max}$),
o número de discípulos e de formas da técnica, respectivamente.
Os discípulos são numerados de $1$ a $N$, e as formas são numeradas de $1$ a
$F$.

As próximas $N$ linhas contém $F$ caracteres cada. O $j$-ésimo caracter da
$i-$ésima linha é \texttt{S} se o discípulo $i$ domina a forma $j$, ou
\texttt{N} se não domina.

%- endblock

%- block output
Imprima uma linha com o número de maneiras de
formar a dupla perfeita.
%- endblock

%- block notes
No primeiro exemplo, a única dupla que pode ser formada é a dos discípulos
$\{1,2\}$.

No segundo exemplo, as possíveis duplas são
$\{1,2\}$, $\{1,4\}$, $\{2,3\}$ e $\{2,4\}$.

%- endblock

%- block editorial
Uma solução direta seria iterar em todos os $O(N^2)$ pares de discípulos e
verificar, em $O(F)$, se cada par domina ao menos uma forma em, totalizando $O(N^2 \times F)$.
Entretanto, esta solução não é rápida o bastante para os limites dados no problema.

Para reduzir a complexidade, note que é possível converter a string dada para
cada discípulo em um \textit{bitmask} de $F$ bits: um inteiro onde o $i-$ésimo
bit de sua representação binária é 1 se o $i-$ésimo caractere é \texttt{S}, ou 0
se é \texttt{N}. Como $F \leq 10$, este inteiro será no máximo $2^{10} - 1 =
1023$, que pode ser armazenado em uma variável \texttt{int}. Esta conversão
é feita em $O(F)$ para cada discípulo, totalizando $O(NF)$.

Seja $bm_i$ o \textit{bitmask} do discípulo $i$. Para testar em $O(1)$ se os discípulos
$i$ e $j$ podem ser uma dupla, basta verificar se $bm_i | bm_j = 2^F-1$ (pois
        $|$ é o operador \textit{ou} bit-a-bit, e
        $2^F-1$ é o \textit{bitmask} com todos os bits em 1). Assim, a
complexidade cai para $O(N^2)$. Entretanto, esta complexidade ainda não é rápida
o bastante.

Note que há no máximo $2^F$ \textit{bitmasks} possíveis, que é no máximo $2^{10} =
1024$ para os limites do problema!

Pré-compute $Q[bm]$, a quantidade de discípulos cuja \textit{bitmask} é $bm$.
Note que, para cada par de \textit{bitmasks} $bm_A$ e $bm_B$ com $bm_A \neq
bm_B$ onde $bm_A | bm_B = 2^F - 1$,
há $Q[bm_A] \times Q[bm_B]$ duplas possíveis de serem formadas.
Assim, itere entre os $O((2^F)^2)$ pares de \textit{bitmasks} distintas
e incremente a resposta em $Q(bm_A) \times Q(bm_B)$ para cada par possível.

Há um \textit{corner case}, que é considerar quando $bm_A$ e $bm_B$ são a mesma
\textit{bitmask}. Note que o único caso em que é possível formar duplas com uma
\textit{bitmask} e ela mesma é a \textit{bitmask} $2^F-1$ (discípulos que
        dominam todas as técnicas). Para contar este caso, incremente a resposta
em $(Q[2^F-1] \times (Q[2^F-1] - 1))/2$, o número de duplas que podem ser formadas
entre eles.

A resposta pode não caber em um inteiro de 32 bits. Use \textit{long long}.

Complexidade total: $O(NF + (2^F)^2)$.

%- endblock
