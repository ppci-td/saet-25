%- block legend
Eduardo é um grande fã de jogos eletrônicos de automação. Recentemente, ele descobriu um novo jogo chamado \textit{Genectorio}. Nesse jogo, o jogador recebe um lado de um filamento de DNA com $N$ nucleotídeos, que podem ser representados pelas letras A, C, G e T.

O objetivo do jogo é construir a sequência complementar do DNA, isto é, para cada nucleotídeo:
\begin{multicols}{2}
\begin{itemize}
  \item A deve ser pareado com T;
  \item T deve ser pareado com A;
  \item C deve ser pareado com G;
  \item G deve ser pareado com C.
\end{itemize}
\end{multicols}
\vspace{-0.5cm}
Entretanto, para deixar o desafio mais interessante, o jogo possui um modo secreto onde a sequência complementar é criptografada utilizando uma variação da \textit{Cifra de César}. A Cifra de César consiste em substituir cada letra da string pela letra que ocorre $k$ posições depois dela no alfabeto (circular), onde $k$ é o \textit{deslocamento} da letra.

Nesse modo, o deslocamento de cada nucleotídeo é determinado por operações de
\texttt{XOR} bit a bit ($\oplus$) da seguinte forma:
\begin{itemize}
  \item Seja $X$ um número inteiro dado na entrada.
  \item Para o primeiro nucleotídeo $N_1$, o deslocamento é calculado como $(X \oplus N_1)$.
  \item Para o segundo nucleotídeo $N_2$, o deslocamento é calculado como $(X \oplus N_1 \oplus N_2)$.
  \item O processo continua seguindo essa lógica cumulativa até o último nucleotídeo.
\end{itemize}

Após aplicar todas as substituições, o jogador deve imprimir a sequência complementar criptografada.

Considere que o alfabeto tem apenas as letras A,C,G e T, nesta ordem. Ainda, as letras A, C, G e T devem ser convertidas para 0, 1, 2 e 3 respectivamente para fins do cálculo do operador \texttt{XOR}.
%- endblock

%- block input
A primeira linha contém dois inteiros $N$ e $X$ ($1 \le N \le 10^5$, $0 \leq X \leq 10^5$).
A segunda linha contém uma \textbf{string} $S$ de tamanho $N$, representando a sequência de nucleotídeos.
\vspace{-0.3cm}
%- endblock

%- block output
Imprima uma linha com a \textbf{string criptografada} resultante após aplicar todas as operações descritas.
\vspace{-0.7cm}
%- endblock

%- block notes
No primeiro exemplo de entrada, a string complementar é \texttt{CGTACG}. O deslocamento da primeira letra é $7 \oplus 2 = 5$, e logo a primeira letra criptografada é C $\rightarrow$ G $\rightarrow$ T $\rightarrow$ A $\rightarrow$ C $\rightarrow$ G.  O deslocamento da segunda letra é $7 \oplus 2 \oplus 1 = 4$, e logo a segunda letra criptografada é G $\rightarrow$ T $\rightarrow$ A $\rightarrow$ C $\rightarrow$ G.
%- endblock

%- block editorial
A solução do problema pode ser divida em 3 etapas:
Primeiro, é necessário compreender que cada nucleotídeo (A, C, G, T) possui um par complementar fixo: A é pareado com T, T com A, C com G e G com C. Podemos representar essas letras por números inteiros \((A=0, C=1, G=2, T=3)\) para facilitar o uso do operador \texttt{XOR}.  

Em seguida, para cada posição \(i\) da string original \(S\), o deslocamento utilizado na criptografia é determinado pelo \texttt{XOR} cumulativo entre \(X\) e os valores inteiros dos nucleotídeos já processados, isto é:
\[
\text{deslocamento}_i = X \oplus S_1 \oplus S_2 \oplus \ldots \oplus S_i
\]
onde \(\oplus\) representa o operador \texttt{XOR} bit a bit.  

Após calcular o deslocamento, ele é reduzido módulo 4 (pois o alfabeto tem apenas 4 letras), e o resultado indica o número de posições que devemos avançar no alfabeto circular \(\{A, C, G, T\}\) a partir da letra complementar do nucleotídeo atual. Assim, a nova letra é obtida por:
\[
\text{reposta} = (\text{complementar}(S_i) + \text{deslocamento}_i) \bmod 4
\]

Complexidade total: $O(N)$.
%- endblock
