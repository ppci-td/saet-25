%- block legend
Betinha da Silva é um estudante que acha arrays de números inteiros algo muito \textit{Pogger} e, por isso, ele coleciona arrays de números inteiros.  
Toda semana Betinha vai na loja de arrays e compra $N$ arrays para adicionar na sua coleção.  

Porém, Betinha é um colecionador muito \textit{RedPill}, portanto ele só coloca em sua coleção um array se ele for um array \textit{Banger}.  

Um array é considerado um array \textit{Banger} se a soma de todos os números nas posições primas do array também for um número primo.  

Betinha acha totalmente \textit{Brutal} ter que verificar se os arrays que ele compra são \textit{Bangers} ou não, então ele compra todos sem mais nem menos.  
Mas às vezes ele pode ser \textit{Moggado} por essa decisão: isso acontece quando ele acaba comprando um conjunto de arrays que não contém nenhum \textit{Banger}, de forma que sobra nada para Betinha colocar em sua coleção naquela semana.  

Sua tarefa é escrever um programa que determine o número de arrays \textit{Bangers} que Betinha conseguiu naquela semana.
%- endblock

%- block input
A primeira linha contém um inteiro $N$ ($1 \leq N \leq 10^3$), o número de arrays que Betinha comprou na loja naquela semana.  

Em seguida, para cada um dos $N$ arrays:  

A primeira linha contém um inteiro $M$ ($1 \leq M \leq 10^3$) , o tamanho do array.  

A linha seguinte contém $M$ inteiros $A$ ($1 \leq A \leq 10^5$), os elementos do array.  
%- endblock

%- block output
A saída deve ser na primeira linha, um inteiro $B$, a quantidade de arrays \textit{Bangers} que Betinha conseguiu naquela semana.  

Na linha seguinte, $B$ inteiros representando as somas correspondentes de cada array \textit{Banger}.  

Caso Betinha tennha sido \textit{Moggado}, ou seja, caso $B = 0$, a saída deve ser a seguinte frase em letras maiúsculas sem aspas: "ITS OVER SOBROU NADA PRO BETINHA"
%- endblock

%- block notes
No notes.
%- endblock

%- block editorial
This is the editorial.
%- endblock
